\documentclass[a4paper]{article}
\author{Chris Swetenham (s1149322)}
\title{IAML Assignment}
\date{November 22, 2011}

\usepackage{fullpage}
\usepackage{verbatim}
\usepackage{enumerate}
\usepackage{graphics}
\usepackage{graphicx}
\usepackage{amsmath}

\begin{document}

\maketitle
\clearpage

\section*{Part A}
\subection*{Question i}
Training from \t{train_faces.arff} with 5-fold CV
NaiveBayes: Accuracy 12\%
J48: Accuracy 37\%
SMO: Accuracy 46\%
\subsection*{Question ii}
AddCluster: EM using 10 clusters corresponding to the 10 subjects in the dataset.
In the histogram for the cluster attribute, we notice that there are two clusters which dominate, with 100 instances each. We were hoping to see 10 roughly equal clusters corresponding to the 10 subjects in the dataset. At this point, checking the histogram of the class labels, we notice that these are unevenly distributed, and further we have 400 instances rather than the 20*10 we were promised. The two dominating classes together sum to 200, the discrepancy between the expected and actual size of the dataset. Finally, taking the clusters and labelling them by class, we see that there is very little correspondance between the two attributes.
The dominating clusters are \i{cluster2} and \i{cluster6}. Taking the first three images from each cluster: [6,12,20] and [2,15,29]:
Comparing this to the first three images from \{cluster1} and \{cluster3}: [1,7,8], [13,17,19]:

We see that the two dominating clusters seem to be pure white and noisy dark images which are not relevant to our task.

\subsection*{Question iii}
Using the filter \t{RemoveWithAttribute}, instances in \i{cluster2} and \{cluster6} were removed. The resulting dataset has 20 instances of each class, as expected from the task description.

\subsection*{Question iv}
After removing the cluster attribute, with 5-fold CV:
NaiveBayes: Accuracy 52.5\%
J48: Accuracy 62\%
SMO: Accuracy 88\%

We can compare this performance with the results from part i) as baseline. After removing the erroneous instances, the accuracy of all three classifiers has greatly improved. Since the dataset was smaller, the time taken to train and evaluate the classfiers was also much shorter.

\subsection*{Question v}
Training SMO classifier from \t{train_faces_clean.arff} with \t{val_faces.arff} as test set.

SMO: Accuracy 92\%

This is comparable to the performance on the training data earlier.

Two example correctly classified instances: [32, 85]
Two example incorrectly classified instances: [15, 42]

The classifier seems to make mistakes on instances where the subject's eyes are entirely in shadow.

\end{document}
